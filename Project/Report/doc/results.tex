\chapter{Results} \label{chap:res}

\section{Analysis of Olympic Games Duration and Evolution}

The Gantt chart in Figure \ref{fig:gantt_olympics} illustrates the scheduling and duration of both Summer and Winter Olympic Games from the inaugural Athens Games in 1896 to the events in 2022. It highlights key milestones and trends in the evolution of the Olympics.

\begin{figure}[ht]
    \centering
    \includegraphics[width=0.8\textwidth]{Gantt Chart of Olympic Games.png}
    \caption{Gantt Chart of Olympic Games Duration}
    \label{fig:gantt_olympics}
\end{figure}

The modern Olympics began in 1896 with exclusively male participants and a limited schedule of events lasting 10 days. By 1900, the Paris Games introduced women's participation and expanded the schedule to over five months, integrating the Olympics into the World’s Fair. This unusual duration reflected the scattered organization and the addition of new sports like golf, tennis, and rowing. Over time, the duration became more standardized, with events typically lasting around two weeks by the mid-20th century, reflecting the Games' growing scale and complexity.

The Winter Olympics were introduced in Chamonix, France, in 1924, marking the start of a separate seasonal competition for winter sports. Notably, France hosted both the Summer and Winter Games in the same year, solidifying its pivotal role in Olympic history.

Additionally, the chart captures regular scheduling patterns, with Summer Games typically held between July and August and Winter Games in February. It also reveals disruptions caused by global conflicts, including cancellations during World War I and World War II.

\section{Chord Diagram for Migration Flow}

Given the increasing rate of immigration over the past few decades, we decided to analyze immigration in the context of the Olympic Games, specifically focusing on Olympic athletes who represent countries different from the ones in which they were born. A key component for this analysis was the inclusion of athletes' birth countries, which were not initially present in the dataset. To address this, we used web scraping to extract athletes' birth countries from their Wikipedia pages and added this information as a new feature to the dataset. Since there have been significant changes in the names and territories of countries over time, we decided to focus only on the last 10 Olympic Games to ensure consistency.

To visualize this analysis, we used a chord diagram. A chord diagram is a graphical tool designed to show relationships between entities, where the flow of migration is represented by the chords. The source and destination countries of the migration are indicated by the base and arrowhead of each chord, respectively. The width of each arc is proportional to the number of athletes migrating from one country to another.

In our chord diagram, the countries are grouped by sub-continent, with a total of 9 sub-continents, plus the Refugee Olympic Team, which is treated as a separate category. Each sub-continent is assigned a unique color theme to improve the clarity of the visualization. To make the analysis more manageable and focused, we only included countries with more than 10 immigrant athletes in the diagram. This filter ensures that the visualization highlights significant migration flows.

\begin{figure}[ht]
    \centering
    \includegraphics[width=0.8\textwidth]{chord_diagram.png}
    \caption{Chord Diagram: Visualizing Olympic Athletes' Migration Flows Since 2010}
    \label{fig:choropleth_map}
\end{figure}

\subsection{Features of the Diagram}
An interactive feature of this diagram is that when hovering over a country's name, additional information is displayed, including the total number of incoming and outgoing immigrants for that country. Additionally, when hovering over a country, the opacity of arrows that do not originate from or end in that country is reduced. This allows the user to focus on immigration flows specifically related to the selected country, highlighting it as the source or destination
\subsection{Insights from the Diagram}
The diagram provides valuable insights into immigration patterns. Most visible immigration flows align closely with political and geographical changes in the world. In particular, highly developed countries like the United States and Britain emerge as major destinations for immigrants, likely due to the better opportunities and higher quality of life they offer.

However, the diagram also reveals patterns specific to the unique nature of the dataset. For instance, Russia was banned from participating in the 2016 Rio and 2020 Tokyo Olympics due to doping violations. In such cases, many Russian athletes opted to compete under other countries' flags just to continue participating in the games.

Another intriguing trend is the movement of athletes between countries in pursuit of better athletic opportunities. For example, some U.S. athletes appear to have moved to Latin American countries, possibly because it was easier to qualify for the Olympic team there. 


\section{Dynamic Choropleths and Bubble Maps}

The dynamic choropleth and bubble maps were designed to provide an interactive exploration of Olympic trends worldwide. These maps enable users to visualize data such as the frequency of Olympic hosting by country, the number of athlete debuts, and medal counts. With dynamic filtering options and multiple visualization styles, they offer a versatile and engaging way to analyze key aspects of Olympic history.

The choropleth map excels at conveying spatial patterns and regional comparisons through color intensity, making it ideal for quickly identifying geographical trends. However, it may obscure information for smaller countries due to their limited map space. In contrast, the bubble map highlights individual data points with proportional circle sizes, ensuring that smaller countries remain visible and providing a more precise representation of numerical values. On the downside, bubble maps can become cluttered in regions with dense data points, potentially making it harder to interpret overlapping bubbles. Together, these visualization styles complement each other, balancing clarity and detail depending on the user's analytical needs.

\begin{figure}[ht]
    \centering
    \includegraphics[width=0.8\textwidth]{Dynamic Choropleth.png}
    \caption{Dynamic Choropleth Map: Visualizing Olympic Hosting Trends}
    \label{fig:choropleth_map}
\end{figure}

\begin{figure}[ht]
    \centering
    \includegraphics[width=0.8\textwidth]{Dynamic Bubble Map.png}
    \caption{Dynamic Bubble Map: Visualizing Olympic Hosting Trends}
    \label{fig:bubble_map}
\end{figure}

\subsection{Features of the Maps}

The maps provide the following functionalities:
\begin{itemize}
    \item \textbf{Visualization Modes:} Users can switch between a choropleth map (Figure \ref{fig:choropleth_map}) and a bubble map (Figure \ref{fig:bubble_map}) for different visual representations of hosting frequency.
    \item \textbf{Filters and Options:} Filtering options include continents, sub-regions, seasons (Summer or Winter), medal types (Gold, Silver or Bronze), if applicable. These allow users to customize the view and focus on specific geographic or temporal trends.
    \item \textbf{Interactivity:} Both maps feature hover-over tooltips displaying detailed information, such as the number of times a country has hosted the Olympics.
\end{itemize}

\subsection{Insights from the Maps}

These maps reveal several insights into Olympic hosting trends:
\begin{itemize}
    \item Countries such as the United States, Great Britain, and France stand out in number of debuts and high hosting frequencies, reflecting their longstanding involvement in the Olympics.
    \item Hosting occurrences are concentrated in Europe and North America, particularly during the early years of the modern Olympics.
    \item The inclusion of filtering options enables the identification of hosting trends by region, season, and medal type, highlighting the geographical expansion of the Games over time.
\end{itemize}

The combination of choropleth and bubble maps exemplifies the power of interactive visualizations, enabling a flexible exploration of the data while revealing key trends and disparities in Olympic hosting.

\section{ADNANE PLOTS}

PLACEHOLDER
