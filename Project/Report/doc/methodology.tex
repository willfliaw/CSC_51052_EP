\chapter{Methodology} \label{chap:method}

The analysis of the Olympic dataset involved a combination of data preprocessing, cleaning, and visualization techniques to uncover meaningful insights. The raw dataset, despite its extensive coverage, required significant preprocessing and cleaning to address inconsistencies and ensure its reliability for analysis. Issues such as duplicate records, missing values, and invalid entries (e.g., unrealistic birth years and medal counts) were identified and carefully resolved. These preprocessing steps laid the foundation for the subsequent analysis, as detailed in the following subsections.

Once the data was cleaned, various visualizations were created to explore patterns and trends in the dataset. Python, with libraries such as \texttt{pandas} and \texttt{matplotlib}, was used to generate preliminary exploratory plots. These static visualizations provided insights into data distributions, temporal trends, and relationships between variables. Examples include boxplots for identifying invalid birth years, line charts for visualizing medal accumulation over time, and scatterplots for exploring correlations.

For more advanced and interactive visualizations, JavaScript and the D3.js library were employed. These tools facilitated the creation of dynamic visualizations, such as choropleth maps for visualizing medal distributions across countries and flow diagrams for illustrating athlete migration patterns. The interactive visualizations incorporated features such as tooltips, zooming, and filtering, allowing users to explore the data in depth and tailor their analysis to specific interests.

This combination of Python for initial exploratory analysis and JavaScript with D3.js for dynamic and interactive visualizations ensured a comprehensive approach to analyzing and presenting the data. Together, these tools provided both high-level summaries and detailed explorations, enhancing the overall analytical process and user experience.

\section{Inconsistent Data Entries}

The raw dataset contained several inconsistencies in data entries that required attention during the preprocessing phase. Two significant issues encountered were duplicate records and invalid birth years. These problems had to be resolved to ensure the dataset's reliability and accuracy for analysis.

\subsection{Duplicate Records}

Duplicate entries for athletes and events were present in the dataset, leading to redundancies and inaccuracies. For instance, multiple records existed for the same athlete across different events or Olympic appearances, making it challenging to analyze unique participation trends. These duplicate entries were systematically identified using key attributes, such as athlete names, event details, and medal information, and subsequently removed to maintain data integrity.

\subsection{Invalid Birth Years}

Another major inconsistency was the presence of invalid or missing birth years for athletes. Implausible values, such as negative birth years or years indicating extreme ages (e.g., over 150 years), were identified in the dataset. These anomalies introduced inaccuracies in calculating athletes' ages at their first Olympic appearance and other related analyses.

To address this, reasonable age boundaries for Olympic participation were established based on historical records. According to sources, the youngest known Olympian was 10 years old \cite{youngest_olympian}, and the oldest recorded Olympian was 73 years old \cite{oldest_olympian}. Using these references, red shaded regions in the visualizations highlight implausible age values falling outside this range.

Figure \ref{fig:birth_years} shows the distribution of athletes' birth years by their first Olympic appearance, with anomalies clearly visible as athletes cannot have their first Olympic appearance before the year their were born. The red shaded region highlights implausible values, including negative ages and values exceeding typical human lifespans. Similarly, Figure \ref{fig:athlete_ages} illustrates the distribution of athletes' ages at their first Olympic appearance.

\begin{figure}[ht]
    \centering
    \includegraphics[width=\textwidth, keepaspectratio]{Distribution of Athletes' Birth Years by First Olympic Appearence.png}
    \caption{Distribution of Athletes' Birth Years by First Olympic Appearance}
    \label{fig:birth_years}
\end{figure}

\begin{figure}[ht]
    \centering
    \includegraphics[width=\textwidth, keepaspectratio]{Distribution of Athletes' Ages at Their First Olympic Appearence.png}
    \caption{Distribution of Athletes' Ages at Their First Olympic Appearance}
    \label{fig:athlete_ages}
\end{figure}

\section{Ambiguous Geographical Data}

Geographical data in the raw dataset was often ambiguous or inconsistent. Key issues included:

\begin{itemize}
    \item \textbf{Host City and Country Mapping:} Host cities were inconsistently labeled or lacked corresponding country information. To resolve this, city names were mapped to their respective countries using geocoding tools such as Nominatim \cite{nominatim}.
    \item \textbf{Country Code Discrepancies:} Standardized two-letter (ISO 3166-1 alpha-2) and three-letter (ISO 3166-1 alpha-3) country codes were assigned to all entries to eliminate inconsistencies in naming conventions.
\end{itemize}

By standardizing geographical data, we ensured consistency and improved the dataset's usability for visualizations involving country-specific analyses.


\section{Missing Metadata}

A significant portion of the dataset contained missing or incomplete metadata, particularly for athletes and events. Common issues included:

\begin{itemize}
    \item \textbf{Incomplete Athlete Information:} Some athletes lacked URLs, full names, or demographic details. Such records were filtered out when critical information was unavailable.
    \item \textbf{Unresolved Medalist Metadata:} Certain medalists had incomplete associations with their events or disciplines, which limited their analytical use.
\end{itemize}

These gaps were addressed where possible, and records that could not be resolved were excluded from further analysis.
