\begin{abstract}

    This report presents a data visualization project focused on analyzing athlete migration, geopolitical influences, and Olympic medal performance during the Olympic Games from 1986 to 2018. Using a combination of datasets, including Olympic medal counts, population data, and geopolitical events, we explore key trends such as migration patterns of athletes, medals per capita, and temporal medal accumulation across countries. The visualizations are designed to reveal the relationship between global political events and athletic representation, as well as how countries perform in relation to their population size. Interactive visualizations, including flow maps, choropleth maps, and line charts, were implemented using D3.js to offer a comprehensive analysis of the data. The project underscores the value of visualizing complex, multi-dimensional datasets to derive meaningful insights into historical and geopolitical factors affecting Olympic performance.

    \vspace{1\baselineskip}

    \textbf{Keywords:} Olympic Games, athlete migration, geopolitical impact, medals per capita, data visualization, D3.js, temporal trends, choropleth maps, interactive visualizations.

\end{abstract}


\begin{summary}

    Ce rapport présente un projet de visualisation de données axé sur l'analyse des migrations d'athlètes, des influences géopolitiques et de la performance des médailles aux Jeux Olympiques de 1986 à 2018. En utilisant une combinaison de jeux de données, incluant les comptes des médailles olympiques, les données de population et les événements géopolitiques, nous explorons les principales tendances telles que les migrations d'athlètes, le nombre de médailles par habitant et l'accumulation temporelle de médailles par pays. Les visualisations sont conçues pour mettre en lumière les relations entre les événements politiques mondiaux et la représentation athlétique, ainsi que les performances des pays en fonction de leur taille démographique. Des visualisations interactives, incluant des flux migratoires, des cartes choroplèthes et des graphiques linéaires, ont été implémentées à l'aide de D3.js afin d'offrir une analyse complète des données. Ce projet souligne l'importance de la visualisation de jeux de données complexes et multidimensionnels pour extraire des informations significatives sur les facteurs historiques et géopolitiques influençant la performance olympique.



    \vspace{1\baselineskip}

    \textbf{Palavras-chave:} Jeux Olympiques, migration des athlètes, impact géopolitique, médailles par habitant, visualisation de données, D3.js, tendances temporelles, cartes choroplèthes, visualisations interactives.

\end{summary}


