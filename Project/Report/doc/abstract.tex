\begin{abstract}

    This report presents a data visualization project focused on analyzing athlete migration and Olympic medal performance during the Olympic Games from 1986 to 2022. Using the Olympic Games Medals 1986-2022 dataset, we explore key trends such as migration patterns of athletes and temporal medal accumulation across countries. The visualizations are designed to reveal patterns in athletic representation and medal distributions over time. Interactive visualizations, including flow maps, choropleth maps, and line charts, were implemented using D3.js to offer a comprehensive analysis of the data. The project underscores the value of visualizing complex, multi-dimensional datasets to derive meaningful insights into historical trends affecting Olympic performance.

    \vspace{1\baselineskip}

    \textbf{Keywords:} Olympic Games, athlete migration, data visualization, D3.js, temporal trends, choropleth maps, interactive visualizations.

\end{abstract}


\begin{secondAbstract}

    Ce rapport présente un projet de visualisation de données axé sur l'analyse des migrations d'athlètes et de la performance des médailles aux Jeux Olympiques de 1986 à 2022. En utilisant le jeu de données Olympic Games Medals 1986-2022, nous explorons les principales tendances telles que les migrations d'athlètes et l'accumulation temporelle de médailles par pays. Les visualisations sont conçues pour mettre en lumière les dynamiques de représentation athlétique et les distributions des médailles au fil du temps. Des visualisations interactives, incluant des flux migratoires, des cartes choroplèthes et des graphiques linéaires, ont été implémentées à l'aide de D3.js afin d'offrir une analyse complète des données. Ce projet souligne l'importance de la visualisation de jeux de données complexes et multidimensionnels pour extraire des informations significatives sur les tendances historiques influençant la performance olympique.

    \vspace{1\baselineskip}

    \textbf{Mots-clés:} Jeux Olympiques, migration des athlètes, visualisation de données, D3.js, tendances temporelles, cartes choroplèthes, visualisations interactives.

\end{secondAbstract}
