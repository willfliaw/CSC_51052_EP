% -----------------------------------------------------------------------------
% chapter.tex
%
% This file defines the chapter and section environments, including:
%   - Custom formatting for chapter titles
%   - Page numbering schemes for chapters
%   - Automatic inclusion of chapters and sections in the Table of Contents
%   - Support for starred chapters/sections that include centered titles
%   - Custom formatting for sections, subsections, and other heading levels
%
% Redefinitions include handling of pretextual chapters, custom TOC entries,
% and formatting of chapter and section headers.
% -----------------------------------------------------------------------------

% Define command to allow optional starred sections
\def\customSectionDefinition#1#2{\@ifstar{\@dblarg{#2}}{\@dblarg{#1}}}

% Redefine \chapter to include custom page style and manage pretextual chapters
\renewcommand\chapter {
    \if@openright\cleardoublepage\else\clearpage\fi
    \thispagestyle{\chapterTitlePageStyle}
    \ifthenelse{\boolean{inPretext}}{
        \ifthenelse{\boolean{afterTOC}} {
            \setboolean{inPretext}{false}
            \BeginTextualPart
        }{}
    }{}
    \global\@topnum\z@
    \@afterindentfalse
    \secdef\@chapter\@schapter
}

% Pretextual chapters (similar to \chapter*, but with special behavior)
\newcommand\pretextualChapter {
    \if@openright\cleardoublepage\else\clearpage\fi
    \thispagestyle{\chapterTitlePageStyle}
    \global\@topnum\z@
    \@afterindentfalse
    \@schapter
}

% Reset subcounters (e.g., sections, subsections) when starting a new chapter
\newcommand{\resetSubCounters}[1]{
    \begingroup
    \let\@elt\@stpelt
    \csname cl@#1\endcsname
    \endgroup
}

% Remove the next chapter or section from the Table of Contents
\newboolean{nextOutOfTOC}
\setboolean{nextOutOfTOC}{false}

% Command to mark the next chapter/section as out of the TOC, with optional mark
\newcommand{\nextMark}{}
\newcommand{\nextOutOfTOC}[1][] {
    \setboolean{nextOutOfTOC}{true}
    \renewcommand{\nextMark}{#1}
}

% Modify how \addcontentsline behaves for chapters/sections not in TOC
\let\oldAddContentsLine\addcontentsline\relax
\newcommand{\customAddContentsLine}[3] {
    \ifthenelse{\boolean{nextOutOfTOC}}
    {\setboolean{nextOutOfTOC}{false}}
    {\ifthenelse{\boolean{hyperrefLoaded}}
    {\addtocontents{#1}{\protect\contentsline{#2}{#3}{\thepage}{#2.\csname the#2\endcsname}}}
    {\oldAddContentsLine{#1}{#2}{#3}}
  }
}

% Custom chapter marking (for headers)
\newcommand{\customChapterMark}[1] {
    \ifthenelse{\boolean{nextOutOfTOC}}
    {\markboth{\nextMark}{\nextMark}}
    {\chaptermark{#1}}
}

% Custom section marking (for headers)
\newcommand{\customSectionMark}[1] {
    \ifthenelse{\boolean{nextOutOfTOC}}
    {\markright{\nextMark}}
    {\sectionmark{#1}}
}

% Redefine \@chapter to manage TOC entry and chapter marking
\def\@chapter[#1]#2 {
    \customChapterMark{#1} % This command must precede addcontentsline
    \ifnum \c@secnumdepth >\m@ne
        \refstepcounter{chapter}
        \typeout{\@chapapp\space\thechapter.}
        \ifthenelse{\boolean{afterTOC}}
        {
            \customAddContentsLine{toc}{chapter}
            {\protect\numberline{\thechapter}#1}
        }{}
    \else
        \ifthenelse{\boolean{afterTOC}}
        {
            \customAddContentsLine{toc}{chapter}{#1}
        }{}
    \fi
    \if@twocolumn\@topnewpage[\@makechapterhead{#2}]
    \else
        \@makechapterhead{#2}\@afterheading
    \fi\par
}

% Redefine \@schapter to behave similarly to \@chapter
\def\@schapter#1 {
    \if@twocolumn
        \@topnewpage[\@makeschapterhead{#1}]
    \else
        \@makeschapterhead{#1}
        \@afterheading
    \fi
    \@mkboth{#1}{#1}
    \ifthenelse{\boolean{includeInTOC}}{
        \ifthenelse{\boolean{afterTOC}}{
            \customAddContentsLine{toc}{chapter}{#1}
        }{}
    }
    {}
    \resetSubCounters{chapter}\par
}

% Redefine \part to manage part numbering, TOC inclusion, and title formatting
\def\@part[#1]#2{
    \ifnum \c@secnumdepth >-2\relax
        \refstepcounter{part}
        \customAddContentsLine{toc}{part}{\protect\numberline{\thepart}#1}
    \else
        \resetSubCounters{part}
        \customAddContentsLine{toc}{part}{#1}
    \fi
    \markboth{}{}{
        \centering
        \interlinepenalty \@M
        \normalfont
        \ifnum \c@secnumdepth >-2\relax
            \partTitleFont\partTitleSize
            \ifthenelse{\boolean{uppercasePartTitles}}{
                \MakeUppercase{\partname~\thepart}
            }
            {\partname~\thepart}
            \par
            \vskip 20\p@
        \fi
        \partTitleFont\partSubtitleSize
        \ifthenelse{\boolean{uppercasePartTitles}}
        {\MakeUppercase{#2}}
        {#2}\par
    }
    \@endpart
}

% Redefine \part* for starred parts without numbering
\def\@spart#1{
    \customAddContentsLine{toc}{part}{#1} % Include part in TOC
    \markboth{}{}
    {\centering
        \interlinepenalty \@M
        \normalfont
        \partTitleFont\partSubtitleSize
        \ifthenelse{\boolean{uppercasePartTitles}}
        {\MakeUppercase{#1}}
        {#1}\par}
    \@endpart
}

% Custom start of sections, handling TOC inclusion and title formatting
\def\customStartSection#1#2#3#4#5#6{
    \if@noskipsec \leavevmode \fi
    \par
    \@tempskipa #4\relax
    \@afterindenttrue
    \ifdim \@tempskipa < \z@
        \@tempskipa - \@tempskipa \@afterindentfalse
    \fi
    \if@nobreak
        \everypar{}
    \else
        \addpenalty\@secpenalty\addvspace\@tempskipa
    \fi
    \@ifstar
    {\customStartStarSection{#1}{#4}{#5}{#6}}
    {\@dblarg{\customStartSectionWithNumber{#1}{#2}{#3}{#4}{#5}{#6}}}
}

% Starred section formatting, with optional inclusion in TOC
\def\customStartStarSection#1#2#3#4#5{
    \@tempskipa #3\relax
    \ifdim \@tempskipa>\z@
        \begingroup #4 {
            \interlinepenalty \@M \centering
            \ifthenelse{\boolean{uppercaseSectionTitles}}
            {\MakeUppercase{#5}}{#5}\@@par
        }
        \endgroup
        \@ifundefined{custom#1mark}{}{\csname custom#1mark\endcsname{#5}}
        \ifthenelse{\boolean{includeInTOC}}
        {\customAddContentsLine{toc}{#1}{#5}}
        {}
    \else
        \def\@svsechd{#4{#5}
            \@ifundefined{custom#1mark}{}{\csname custom#1mark\endcsname{#5}}
            \ifthenelse{\boolean{includeInTOC}}
            {\customAddContentsLine{toc}{#1}{#5}}{}
        }
    \fi
    \@xsect{#3}
}

% Numbered section formatting, with TOC inclusion and number formatting
\def\customStartSectionWithNumber#1#2#3#4#5#6[#7]#8 {
    \ifnum #2 > \c@secnumdepth
        \let\@svsec\@empty
    \else
        \refstepcounter{#1}
        \protected@edef\@svsec{\@seccntformat{#1}\relax}
    \fi
    \@tempskipa #5 \relax
    \ifdim \@tempskipa > \z@
        \begingroup #6 {
            \@hangfrom{\hskip #3\relax\@svsec}
            \interlinepenalty \@M
            \ifthenelse{\boolean{uppercaseSectionTitles}}
            {\MakeUppercase{#8}}{#8}\@@par
        }
        \endgroup
        \@ifundefined{custom#1mark}{}{\csname custom#1mark\endcsname{#7}}
        \customAddContentsLine{toc}{#1}{
            \ifnum #2 > \c@secnumdepth
            \else
                \protect\numberline{\csname the#1\endcsname}
            \fi
            #7
        }
    \else
        \def\@svsechd {
            #6 {
                    \hskip #3 \relax
                    \@svsec \ifthenelse{\boolean{uppercaseSectionTitles}}
                    {\MakeUppercase{#8}}{#8}
                }
            \@ifundefined{custom#1mark}{}{\csname custom#1mark\endcsname{#7}}
            \customAddContentsLine{toc}{#1}{
                \ifnum #2 > \c@secnumdepth
                \else
                    \protect\numberline{\csname the#1\endcsname}
                \fi
                #7
            }
        }
    \fi
    \@xsect{#5}
}

% Custom page number style in TOC as required by NBR 6027
\let\old@dottedtocline\@dottedtocline
\renewcommand{\@dottedtocline}[5]{
    \ifthenelse{\boolean{TOCPageNumStyle}}{
        {
                \renewcommand{\@pnumwidth}{2.5em}
                \renewcommand{\@tocrmarg}{3.5em}
                \old@dottedtocline{#1}{#2}{#3}{#4}
                {\ifthenelse{\equal{#5}{}}{}{p.\thinspace#5}}
            }
    }
    {\old@dottedtocline{#1}{#2}{#3}{#4}{#5}}
}

% Define the font for chapter titles
\ifthenelse{\boolean{uppercasePartTitles}}
{\newcommand{\chapterTitleFont}{\bfseries\upshape}}
{\newcommand{\chapterTitleFont}{\bfseries\itshape}}

% Font for chapter titles in TOC
\newcommand{\TOCChapterFont}{\chapterTitleFont\upshape}

% Define the font for section titles
\newcommand{\sectionTitleFont}{\bfseries}

% Define sizes for chapter and part
\newcommand{\chapterSize}{\normalsize}
\newcommand{\uppercaseChapterSize}{\normalsize}
\newcommand{\titleSize}{\Large}

% Defines how regular chapters (\chapter) are typeset
\def\@makechapterhead#1{
    {
            \normalfont\chapterSize\chapterTitleFont
            \setSpacing{single} % Single line spacing for chapter titles
            \vspace*{\spaceHeaderChapter} % Space above the chapter title
            \noindent
            \parbox[b]{\textwidth}{
                \parbox[t]{4ex}{\thechapter} % Space between the chapter number and title
                \parbox[t]{\textwidth-4ex-1ex}{
                    \interlinepenalty\@M\raggedright
                    \ifthenelse{\boolean{uppercasePartTitles}}
                    {\MakeUppercase{#1}}
                    {#1}
                }
                \vspace*{\spaceAfterChapter} % Space below the chapter title
            }\\[0pt]\vspace{\offsetAfterChapter} % Offset for additional space
        }
}

% Flag for alternate chapter title formatting
\newif\ifotherhead
\otherheadfalse

% Defines how starred chapters (\chapter*) are typeset
\def\@makeschapterhead#1{
    \ifotherhead
        % Alternate format for starred chapter titles
        \vspace*{-1.2cm}\par {
            \centering\normalfont\chapterSize\chapterTitleFont%
            \ifthenelse{\boolean{uppercasePartTitles}}
            {\MakeUppercase{#1}}
            {#1}
            \par
        }
        \vspace{45pt}
        \par
        \otherheadfalse % Reset flag
    \else
        \vspace*{0pt}\par {
            \centering\normalfont\chapterSize\chapterTitleFont
            \ifthenelse{\boolean{uppercasePartTitles}}
            {\MakeUppercase{#1}}
            {#1}
            \par
        }
        \vspace{45pt}
        \par
    \fi
}

% Redefine \l@chapter to apply page number styles in the TOC
\renewcommand\l@chapter[2]{
    \ifnum \c@tocdepth >\m@ne
        \addpenalty{-\@highpenalty}
        \vskip 1.0em \@plus\p@
        \setlength\@tempdima{1.5em}
        \begingroup
        \ifthenelse{\boolean{TOCPageNumStyle}}
        {\renewcommand{\@pnumwidth}{3.5em}}{}
        \parindent \z@ \rightskip \@pnumwidth
        \parfillskip -\@pnumwidth
        \leavevmode \normalsize\TOCChapterFont
        \advance\leftskip\@tempdima
        \hskip -\leftskip
        #1\nobreak\hfil \nobreak
        \ifthenelse{\boolean{TOCPageNumStyle}}
        {\hb@xt@\@pnumwidth{\hss\ifthenelse{\not\equal{#2}{}}{{\normalfont p.\thinspace#2}}{}}\par}
        {\hb@xt@\@pnumwidth{\hss #2}\par}
        \penalty\@highpenalty
        \endgroup
    \fi
}

% Redefine \part to manage part numbering, TOC inclusion, and title formatting
\renewcommand\part{
    \if@openright\cleardoublepage\else\clearpage\fi
    \thispagestyle{\chapterTitlePageStyle}
    \if@twocolumn\onecolumn\@tempswatrue\else\@tempswafalse\fi
    \null\vfil\secdef\@part\@spart
}

% Redefine \l@part to apply custom page number styles in the TOC
\renewcommand\l@part[2]{
    \ifnum \c@tocdepth  > -2 \relax
        \addpenalty{-\@highpenalty}
        \addvspace{2.25em \@plus\p@}
        \begingroup
        \parindent \z@ \rightskip \@pnumwidth
        \parfillskip -\@pnumwidth
        \noindent{
            \leavevmode
            \TOCChapterFont\large\noindent%
            #1\hfil \hb@xt@\@pnumwidth{\hss #2}
        }
        \par
        \nobreak
        \global\@nobreaktrue
        \everypar{\global\@nobreakfalse\everypar{}}
        \endgroup
    \fi
}

% Redefine \section to apply custom styles and TOC inclusion
\renewcommand\section{
    \customStartSection{section}{1}{\z@}
    {-3.5ex \@plus -1ex \@minus -.2ex}
    {2.3ex \@plus.2ex}
    {\setSpacing{single}\normalfont\sectionTitleFont\normalsize}
}

% Redefine \subsection with custom styles
\renewcommand\subsection{
    \customStartSection{subsection}{2}{\z@}
    {-3.25ex \@plus -1ex \@minus -.2ex}
    {1.5ex \@plus .2ex}
    {\setSpacing{single}\normalfont\sectionTitleFont\normalsize}
}

% Redefine \subsubsection with custom styles
\renewcommand\subsubsection{
    \customStartSection{subsubsection}{3}{\z@}
    {-3.25ex \@plus -1ex \@minus -.2ex}
    {1.5ex \@plus .2ex}
    {\setSpacing{single}\normalfont\sectionTitleFont\normalsize}
}

% Redefine \paragraph with custom styles
\renewcommand\paragraph{
    \customStartSection{paragraph}{4}{\z@}
    {3.25ex \@plus1ex \@minus.2ex}
    {-1em}
    {\setSpacing{single}\normalfont\sectionTitleFont\normalsize}
}

% Redefine \subparagraph with custom styles
\renewcommand\subparagraph{
    \customStartSection{subparagraph}{5}{\parindent}
    {3.25ex \@plus1ex \@minus .2ex}
    {-1em}
    {\setSpacing{single}\normalfont\sectionTitleFont\normalsize}
}
